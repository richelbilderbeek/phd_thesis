\chapter*{Samenvatting}
\addcontentsline{toc}{chapter}{Samenvatting}

{\selectlanguage{dutch}

% What is the thesis about?
% What is the purpose of the thesis?
% What were the methods used to research the information?
% What are the results, conclusions, and recommendations that the thesis presents?


\noindent 
\dropcap{S}{oortvorming} is een biologisch proces waarbij
nieuwe soorten ontstaan.
Zo was er ooit een diersoort die de voorouder zou worden
van alle aapachtigen.

Al het leven op Aarde stamt af van een gezamelijke voorouder.
Met een fylogenetische boom 

In hoofdstuk 2 laat ik \verb;babette; zien: een R package waarmee je BEAST2, 
een Bayesiaans phylogenetisch inferentieprogramma, kunt aanroepen.

Vraag is: wat is de invloed van het soortvormingsmodel op de inferentie?

In hoofdstuk 3 laat ik een methode zien hoe je kunt meten hoe groot deze invloed is.

In hoofdstuk 4 laat ik zien wat de fout is die we maken in onze inferentie als we
een soortvormingsmodel gebruiken waarin twee soorten tegelijkertijd kunnen ontstaan.

In hoofdstuk 5 laat ik zien wat de fout is die we maken in onze inferentie als we
een soortvormingsmodel gebruiken waarin soortvorming tijd kost.

Het mooie aan mijn onderzoek is dat andere wetenschappers er zelf gemakkelijk ook wat mee kunnen:


} % ~\selectlanguage{dutch}
