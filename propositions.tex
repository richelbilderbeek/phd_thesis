\documentclass{dissertation}

%% Turn off page numbering for the propositions and make the margins on both
%% sides equal and symmetrical.
\geometry{twoside=false}
\pagestyle{empty}

\begin{document}

%% Specify the title and author of the thesis. This information will be used on
%% both the English and Dutch side and in the metadata of the final PDF.
\title[Optional Subtitle]{Title}
\author{Albert}{Einstein}

\begin{center}

{\Large\titlefont\bfseries Propositions}

\bigskip

accompanying the dissertation

\bigskip

%% Print the title.
{\makeatletter
\titlestyle\bfseries\large\@title
\makeatother}

%% Print the optional subtitle.
{\makeatletter
\ifx\@subtitle\undefined\else
    \titlefont\titleshape\@subtitle
\fi
\makeatother}

\bigskip

by

\bigskip

%% Print the full name of the author.
\makeatletter
{\large\titlefont\bfseries\@firstname\ {\titleshape\@lastname}}
\makeatother

\end{center}

\bigskip
\bigskip

\begin{enumerate}

\item Proposition 1.
\item Sampling the most recent common representative from an incipient
  species tree, does not always result in a species tree with shortest
  branch lengths
\item Selecting a teacher for an academic programming course should not be done on his/her academic merit, but by his/her programming skill instead.
\item A critical mind destroys delusions and having something to say at funerals
\item Telling a PhD to hurry is as useful as telling a PhD to follow the laws of physics
\item Propositions are like women
\item Fools ignore complexity. Pragmatists suffer it. Some can avoid it. Geniuses remove it. Alan Perlis
\item Phylogenetic trees are like women
\item Science means never having to say 'Trust me'. Simine Vazire 
\item Men are like women

\end{enumerate}

\bigskip
\bigskip

%% Apart from the name and title of the supervisor, the following text is
%% dictated by the promotieregelement.
\begin{center}
These propositions are regarded as opposable and defendable, and have been approved as such by the promotor prof.\ dr.\ A.\ Kleiner.
\end{center}

\clearpage
{\selectlanguage{dutch}

\begin{center}

{\Large\titlefont\bfseries Stellingen}

\bigskip

behorende bij het proefschrift

\bigskip

%% Print the title.
{\makeatletter
\titlestyle\bfseries\large\@title
\makeatother}

%% Print the optional subtitle.
{\makeatletter
\ifx\@subtitle\undefined\else
    \titlefont\titleshape\@subtitle
\fi
\makeatother}

\bigskip

door

\bigskip

%% Print the full name of the author.
\makeatletter
{\large\titlefont\bfseries\@firstname\ {\titleshape\@lastname}}
\makeatother

\end{center}

\bigskip
\bigskip

\begin{enumerate}

\item Stelling 1.
\item Stelling 2.
\item Stelling 3.
\item Stelling 4.
\item Stelling 5.
\item Stelling 6.
\item Stelling 7.
\item Stelling 8.
\item Stelling 9.
\item Stelling 10.

\end{enumerate}

\bigskip
\bigskip

%% Apart from the name and title of the supervisor, the following text is
%% dictated by the promotieregelement.
\begin{center}
Deze stellingen worden opponeerbaar en verdedigbaar geacht en zijn als zodanig goedgekeurd door de promotor prof.\ dr.\ A.\ Kleiner.
\end{center}

}

\end{document}

