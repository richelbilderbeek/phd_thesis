\documentclass{dissertation}

%% Turn off page numbering for the propositions and make the margins on both
%% sides equal and symmetrical.
\geometry{twoside=false}
\pagestyle{empty}

\begin{document}

%% Specify the title and author of the thesis. This information will be used on
%% both the English and Dutch side and in the metadata of the final PDF.
\title[Optional Subtitle]{Title}
\author{Albert}{Einstein}

\begin{center}

{\Large\titlefont\bfseries Propositions}

\bigskip

accompanying the dissertation

\bigskip

%% Print the title.
{\makeatletter
\titlestyle\bfseries\large\@title
\makeatother}

%% Print the optional subtitle.
{\makeatletter
\ifx\@subtitle\undefined\else
    \titlefont\titleshape\@subtitle
\fi
\makeatother}

\bigskip

by

\bigskip

%% Print the full name of the author.
\makeatletter
{\large\titlefont\bfseries\@firstname\ {\titleshape\@lastname}}
\makeatother

\end{center}

\bigskip
\bigskip

\begin{enumerate}

\item A tool that allows one to do reproducible science from a script
      benefits the scientific field as a whole (this thesis, chapter 2)
\item Quantifying the error we make in phylogenetic inference is
      only a first step towards rejecting speciation models that are
      unnecessarily complex (this thesis, chapter 3)
\item It is defensible to use a Yule tree prior when the tree
      is generated by a BD process, when using the DNA sequences
      of only extant species (this thesis, chapter 3, figure 3.15)
\item It is defensible to use a BD tree prior when the tree
      is generated by an MBD process, when using the DNA sequences
      of only extant species (this thesis, chapter 4)
\item Sampling the most recent common representative from an incipient
      species tree, does not always result in a species tree with shortest
      branch lengths (this thesis, figure 5.4)
\item Open Science is a pleonasm
\item When you assume that your computational experiment will not be repeated,
      then you can safely assume your computational experiment will not be repeated
\item If one yells at a computer, one should also say sorry
\end{enumerate}


% Revoved ones:
% \item Beginners enjoy programming languages in which it is easy to do something.
%       Experts prefer programming languages in which it is hard to do the incorrect thing.
% \item Everything is easier said than done, except talking, which is about the same (Anon)
% \item Disabling tests does not make a bug go away
% \item One of the most satisfactory activity in programming, is to delete code
% \item Talk is cheap. Show me the code. Linus Torvalds
% \item The most important thing a scientist needs is a place to think, while
%       being alone and undisturbed. Ironically, such places are very rare
%       within the university
% \item Telling a PhD to hurry is as useful as telling a PhD to follow the laws of physics
% \item Phylogenetic trees are like women
% \item Propositions are like women: X
% \item X are like women: X
% \item Y are like women: X
% \item Men are like women: X
% \item Quoting wise men does not a wise man maketh.
% \item It is beneficial for an academic career if people think you are smart.
%       It is detrimental for a scientific career to have known programming bugs.
%       It is neutral for a scientific career to expose bugs by writing software tests.
%       Writing obfuscated code is a way to hide bugs and makes one look smart to beginner programmers.
%       Therefore, it is beneficial for an academic career to write obfuscated code,
%       as long as writing software tests is only neutral.
% \item Hiding bugs by writing obfuscated code is both delusional as making you look smart to beginner programmers.
% \item Expecting a bug to go away on its own is a behavior than can be classified as beginner or delusional
% \item Programming is simple in the same way as running a marathon is simple
% \item Science means never having to say 'Trust me'. Simine Vazire
% \item Fools ignore complexity. Pragmatists suffer it. Some can avoid it. Geniuses remove it. Alan Perlis
% \item A critical mind destroys delusions and having something to say at funerals

\end{enumerate}

\bigskip
\bigskip

%% Apart from the name and title of the supervisor, the following text is
%% dictated by the promotieregelement.
\begin{center}
These propositions are regarded as opposable and defendable, and have been approved as such by the promotor prof.\ dr.\ R.S.\ Etienne.
\end{center}

\clearpage
{\selectlanguage{dutch}

\begin{center}

{\Large\titlefont\bfseries Stellingen}

\bigskip

behorende bij het proefschrift

\bigskip

%% Print the title.
{\makeatletter
\titlestyle\bfseries\large\@title
\makeatother}

%% Print the optional subtitle.
{\makeatletter
\ifx\@subtitle\undefined\else
    \titlefont\titleshape\@subtitle
\fi
\makeatother}

\bigskip

door

\bigskip

%% Print the full name of the author.
\makeatletter
{\large\titlefont\bfseries\@firstname\ {\titleshape\@lastname}}
\makeatother

\end{center}

\bigskip
\bigskip

\begin{enumerate}

\item Stelling 1.
\item Stelling 2.
\item Stelling 3.
\item Stelling 4.
\item Stelling 5.
\item Stelling 6.
\item Stelling 7.
\item Stelling 8.
\item Stelling 9.
\item Stelling 10.

\end{enumerate}

\bigskip
\bigskip

%% Apart from the name and title of the supervisor, the following text is
%% dictated by the promotieregelement.
\begin{center}
Deze stellingen worden opponeerbaar en verdedigbaar geacht en zijn als zodanig goedgekeurd door de promotor prof.\ dr.\ A.\ Kleiner.
\end{center}

}

\end{document}

