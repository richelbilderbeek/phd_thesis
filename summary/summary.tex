\chapter*{Summary}
\addcontentsline{toc}{chapter}{Summary}

[This is just a Google Translate, will translate it if Dutch version is OK]

Speciation is a biological process that creates new species. Right away
Phylogenetic tree allows you to see the evolutionary history and kinship of species
view. These trees cannot be measured directly. Be instead
they calculated. Behind this calculation is a mathematical model with many assumptions,
including the assumption how speciation works.
In that test piece I look at the effect of assuming how speciation works.
We think that the most commonly used speciation models are simple enough,
but not too simple. I measure to what extent that is correct.
In Chapter 2 I showed Babette: an R package that allows you to use BEAST2, a Bayesi
can call on phylogenetic information program. babette is a flexible and
become a robust program.
In chapter 3 I showed pirouette: an R package that you can use to measure how
is the influence of a correct or incorrect speciation model. Me and my co-author
show that this instrument works properly, provided you do enough repetitions
chance effects play a small role.
In chapters 4 and 5 I showed the influence of using a standard
speciation model if the real speciation process is somewhat more complex. In
Chapter 4 the real phylogenetic trees are simulated with a speciation
process in which two types can arise simultaneously. Me and my co-author
to show that ...
In Chapter 5, the real phylogenetic trees are simulated with a species
process in which speciation does take time. I show that ... and that it's the effect
from [sampling] ... is.
This dissertation teaches us that ...
The great thing about my research is that other scientists find it easy themselves
also something that can go along: both babette and pirouette are flexible and professional R
packages. babette has made it possible to conduct more extensive research
phylogenetic models, because now an experiment can be done from a script
instead of manually setting each experiment. With pirouette can
scientists finally measure in a standard way to what extent a more complex one
species formation model is worth using.
